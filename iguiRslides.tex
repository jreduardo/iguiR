%%=============================================================================

\documentclass[10pt, aspectratio=169, 
    serif, mathserif, professionalfont, table, svgnames]{beamer}

%%-----------------------------------------------------------------------------
%% Incluídos pelo knitr.

\usepackage[]{graphicx}
\usepackage[]{color}
%% maxwidth is the original width if it is less than linewidth
%% otherwise use linewidth (to make sure the graphics do not exceed the margin)
\makeatletter
\def\maxwidth{ %
  \ifdim\Gin@nat@width>\linewidth
  \linewidth
  \else
  \Gin@nat@width
  \fi
}
\makeatother

\definecolor{fgcolor}{rgb}{0.345, 0.345, 0.345}
%\definecolor{fgcolor}{rgb}{0.5, 0.5, 0.5}
\newcommand{\hlnum}[1]{\textcolor[rgb]{0.686,0.059,0.569}{#1}}%
\newcommand{\hlstr}[1]{\textcolor[rgb]{0.192,0.494,0.8}{#1}}%
\newcommand{\hlcom}[1]{\textcolor[rgb]{0.678,0.584,0.686}{\textit{#1}}}%
\newcommand{\hlopt}[1]{\textcolor[rgb]{0,0,0}{#1}}%
\newcommand{\hlstd}[1]{\textcolor[rgb]{0.345,0.345,0.345}{#1}}%
\newcommand{\hlkwa}[1]{\textcolor[rgb]{0.161,0.373,0.58}{\textbf{#1}}}%
\newcommand{\hlkwb}[1]{\textcolor[rgb]{0.69,0.353,0.396}{#1}}%
\newcommand{\hlkwc}[1]{\textcolor[rgb]{0.333,0.667,0.333}{#1}}%
\newcommand{\hlkwd}[1]{\textcolor[rgb]{0.737,0.353,0.396}{\textbf{#1}}}%

\usepackage{framed}
\makeatletter
\newenvironment{kframe}{%
  \def\at@end@of@kframe{}%
  \ifinner\ifhmode%
  \def\at@end@of@kframe{\end{minipage}}%
\begin{minipage}{\columnwidth}%
  \fi\fi%
  \def\FrameCommand##1{\hskip\@totalleftmargin \hskip-\fboxsep
    \colorbox{shadecolor}{##1}\hskip-\fboxsep
    % There is no \\@totalrightmargin, so:
    \hskip-\linewidth \hskip-\@totalleftmargin \hskip\columnwidth}%
  \MakeFramed {\advance\hsize-\width
    \@totalleftmargin\z@ \linewidth\hsize
    \@setminipage}}%
{\par\unskip\endMakeFramed%
  \at@end@of@kframe}
\makeatother

\definecolor{shadecolor}{rgb}{.97, .97, .97}
\definecolor{messagecolor}{rgb}{0, 0, 0}
\definecolor{warningcolor}{rgb}{1, 0, 1}
\definecolor{errorcolor}{rgb}{1, 0, 0}
\newenvironment{knitrout}{}{} % an empty environment to be redefined in TeX

\usepackage{alltt}

%% Tamanho de fonte e distância entre linhas.
\renewenvironment{knitrout}{
  \footnotesize\renewcommand{\baselinestretch}{0.75}
}{}

%%-----------------------------------------------------------------------------
%% Pacotes padrões.

%% Fontes.
\usepackage{palatino}
\usepackage{eulervm}
\usepackage[none]{ubuntu}
\renewcommand{\ttdefault}{ubuntumono}
\renewcommand{\ttfamily}{\fontUbuntuMono}

%% Verbatim na fonte Ubuntu Mono.
\usepackage{verbatim}
\makeatletter
\def\verbatim@font{\small\fontUbuntuMono}
\makeatother

% Esses pacotes dão clash.
% http://tex.stackexchange.com/questions/51488/option-clash-with-xcolor-and-tikz
\usepackage{xcolor} %% opções no \documentclass{} para evitar clash.
\definecolor{mycolor}{rgb}{0.13,0.53,0.53}
\definecolor{mycolor2}{rgb}{0.725,0,0.18}

% http://tex.stackexchange.com/questions/50747/options-for-appearance-of-links-in-hyperref
\usepackage{hyperref}
\hypersetup{colorlinks, allcolors=.,
  urlcolor=mycolor2, runcolor=orange}

\usepackage[brazil]{babel}
\usepackage[utf8]{inputenc}
\usepackage{graphicx}
\usepackage{amsmath, amsfonts, amssymb, amsxtra, amsthm, icomma}
\usepackage{geometry, calc, setspace, indentfirst}
\usepackage{multicol, multirow}
\usepackage{pbox}
\usepackage{colortbl}
\usepackage{booktabs} % \toprule, \midrule and \bottomrule em tabelas.
\usepackage{enumerate}
% \usepackage{paralist} %% compact{item,enum}. Não usar em beamer.
\usepackage{float}
\usepackage[hang]{caption}
\captionsetup{font=footnotesize, labelfont=footnotesize, labelsep=period}
\usepackage{fancybox}
\usepackage{fancyvrb} %% Permite vebatim dentro de fbox{}.

%% Texto no corpo do beamer justificado.
\usepackage{ragged2e}
\justifying

%%-----------------------------------------------------------------------------

%% A lot of options: http://latex-community.org/forum/viewtopic.php?f=55&t=17646
\useoutertheme[width=60pt, height=30pt, right, hideothersubsections]{sidebar}
\setbeamercolor{structure}{fg=mycolor}

\makeatletter
\setbeamertemplate{section in toc}[sections numbered]
\setbeamertemplate{subsection in toc}[subsections numbered]
\setbeamertemplate{sections/subsections in toc}[ball]{}
\setbeamertemplate{section in sidebar right}[sections numbered]
\setbeamertemplate{caption}[numbered]
\setbeamertemplate{frametitle continuation}{\gdef\beamer@frametitle{}}
\setbeamertemplate{navigation symbols}{} %% Retira a barra de navegação.
% \setbeamertemplate{blocks}[rounded][shadow=FALSE]
% \setbeamercolor{block title}{fg=structure, bg=mycolor!20!white}
\makeatother

% http://tex.stackexchange.com/questions/109816/automatic-frame-titles-subtitles-with-condition
\makeatletter
\CheckCommand*\beamer@checkframetitle{%
  \@ifnextchar\bgroup\beamer@inlineframetitle{}}
\renewcommand*\beamer@checkframetitle{%
  \global\let\beamer@frametitle\relax\@ifnextchar%
  \bgroup\beamer@inlineframetitle{}}
\makeatother

\addtobeamertemplate{frametitle}{
  \ifx\insertframetitle\empty
    \ifx\insertframesubtitle\empty
      \ifx\insertsubsection\empty 
        \frametitle{\insertsectionhead}
      \else
        \frametitle{\insertsubsectionhead}\framesubtitle{\insertsectionhead}
      \fi  
    \else     
    \fi  
  \else
  \fi
}{}

%%-----------------------------------------------------------------------------

% \usebackgroundtemplate{
%   \includegraphics[width=\paperwidth]{./images/ufpr_bg3.jpg}
% }

\AtBeginSection[]{
  \begin{frame}[c, allowframebreaks]
    \frametitle{\phantom{1}}
    \framesubtitle{\phantom{1}}
    \begin{center}
      \textcolor{mycolor}{\thesection} \\ \vspace{0.3cm}
      \parbox{0.6\textwidth}{
        \centering \textcolor{mycolor}{\LARGE \insertsection}}\\
    \end{center}
  \end{frame}
}

%%-----------------------------------------------------------------------------
%% Definições dos proprietários.

\title[Explorando interfaces gráficas com o R]{\LARGE Explorando interfaces gráficas com o R}
\author[]{\small
  Prof. Dr. Walmes M. Zeviani\\
  Eduardo E. Ribeiro Jr\\ 
  Vanessa F. Sehaber\\
  Karina B. Rebuli\\
  Henrique A. Laureano
}

\institute[UFPR]{
  Laboratório de Estatística e Geoinformação \\
  Programa de Educação Tutorial\\
  Departamento de Estatística \\
  Universidade Federal do Paraná}
\date{}
% \logo{\includegraphics[width=1.4cm]{./images/ufpr_logo.jpg}}

%%=============================================================================

\begin{document}

\frame{
  \vspace{-1.0cm}
  \titlepage
  \vspace{-1.6cm}
  \begin{center}
    % \includegraphics[width=4.0cm]{./images/leg2.png}\\
    \href{http://www.leg.ufpr.br}{www.leg.ufpr.br}\\
	%\href{http://www.pet.est.ufpr.br}{www.leg.ufpr.br}\\    
	\texttt{walmes@ufpr.br}
  \end{center}
}

\begin{frame}{Conteúdo}
  %\begin{multicols}{2}
    \small{\tableofcontents}
  %\end{multicols}
\end{frame}

%%=============================================================================

\section{Introdução}

\subsection{Motivação}

\begin{frame}

  \begin{itemize}
  \item Se uma imagem vale mais que 1000 palavras, então um recurso
    interativo vale mais que 1000 imagens
  \item Oferecer recursos para pessoas com menor proeficiência de
    programação
  \item Recursos web não requerem instalação/conhecimento em R
  \end{itemize}

\end{frame}

\subsection{Conteúdo}

\begin{frame}
\vspace{-1.0cm}

\begin{columns}[t]
\begin{column}{.3\textwidth}
\\ \vspace{1cm}
	Recursos interativos
  
  \begin{itemize}
  \setbeamercovered{transparent=35}
    \uncover<2>{
      \item \texttt{animation}
      }
    \uncover<3>{
      \item \texttt{rgl}
      }
    \uncover<4>{
      \item \texttt{googleVis}
      }
    \uncover<5>{
      \item \texttt{gWidgets}
      }
    \uncover<6>{
      \item \texttt{rpanel}
      }
    \uncover<7>{
      \item \texttt{shiny}
      }
     \end{itemize}
     \vspace{1cm}
\end{column}

\begin{column}{.6\textwidth}
\\ \vspace{0.5cm}
    \only<2>{
      {\center
      \includegraphics[scale=0.3]{images/preview_ani}}}
    \only<3>{
      {\center
      \includegraphics[scale=0.3]{images/preview_rgl}}}
    \only<4>{
      {\center 
      \vspace{-0.7cm}
      \includegraphics[scale=0.3]{images/preview_ggvis}}}
    \only<5>{
      {\center
      \includegraphics[scale=0.3]{images/preview_gwidgets}}}
    \only<6>{
      {\center
      \includegraphics[scale=0.3]{images/preview_rpanel}}}
    \only<7>{
      {\center
      \includegraphics[scale=0.3]{images/preview_shiny}}}
\end{column}
    \end{columns}

\end{frame}

\subsection{Recursos interativos básicos}

\begin{frame}

  \begin{itemize}
  \item \texttt{locator()} e \texttt{identify()}
  \item \texttt{readline()}
  \item \texttt{for} e \texttt{while} com \texttt{Sys.sleep()}
  \end{itemize}

\end{frame}

\section{\texttt{animation}}

%----------------------------------------------------------------------

\subsection{Descrição}

\begin{frame}

  \begin{quote}
    To turn ideas in animations (as quick and faithfully as possible).
  \end{quote}

  \texttt{animation} contém funções para produzir
  animações com o R. As animações podem ser produzidas em vários
  formatos: flash, gif, html, pdf e vídeos.

  \begin{itemize}
  \item Autores: Yihui Xie, Lijia Yu, Weicheng Zhu.
  \item Lançamento: 11-Nov-2007.
  \item Versão: 2.3.
  \item URL:
    \url{http://cran.r-project.org/web/packages/animation/index.html},
    \url{http://yihui.name/animation/}
  \item Third-party software: ImageMagik (gif, mpeg convert), SWF Tools
    (\texttt{png2swf}, \texttt{jpeg2swf}, \texttt{pdf2swf})
  \end{itemize}

\end{frame}

%----------------------------------------------------------------------

\begin{frame}
  \begin{itemize}
  \item Na janela gráfica
    \begin{itemize}
    \item Mais natural;
    \item Não requer software extra.
    \end{itemize}
  \item HTML
    \begin{itemize}
    \item Não requer software extra, apenas navegador;
    \item Interface de um player de vídeo com botões de play, pause,
      etc;
    \item Não precisa ter o R, pode usar o Rweb.
    \end{itemize}
  \item GIF
    \begin{itemize}
    \item Requer \texttt{ImageMagick} ou \texttt{GraphicsMagick} para
      converter sequência de imagens em gifs.
    \end{itemize}
  \item Video
    \begin{itemize}
    \item Requer \texttt{FFmpeg} para converter sequência de imagens em
      vídeos.
    \end{itemize}
  \item Flash
    \begin{itemize}
    \item Requer \texttt{SWFTools} para criar animações em flash.
    \end{itemize}
  \end{itemize}
\end{frame}

%----------------------------------------------------------------------

\subsection{Como usar}

\begin{frame}

\input{./tikz/ex_anima1.tex}

\end{frame}

%----------------------------------------------------------------------

\subsection{Exemplos}

\begin{frame}

  Praticando:
  \begin{enumerate}
  \item \href{run:./R/animation/animation.R}{R Script animation}
  \item \href{run:./animation/animation.html}{Galeria animation IGUIR}
  \end{enumerate}
  
  \vspace{0.5cm}
  Algumas aplicações com o animation:
  \begin{itemize}
  \item \href{http://vis.supstat.com/categories.html\#animation-ref}{Galeria
      do autor},
  \item \href{http://www.r-bloggers.com/?s=animation}{Busca no R
      Bloggers}
  \end{itemize}

\end{frame}

\section{\texttt{rgl}}

\subsection*{Descrição}

%----------------------------------------------------------------------

\begin{frame}

  \texttt{rgl} oferece funções de médio a alto nível para gráficos 3D
  interativos, tanto as representações em 3D de gráficos simples em 2D
  quanto para representações objetos geométricos no espaço (cubos,
  elipses, etc). A visualização pode ser na tela com OpenGL ou em outros
  formados como WebGL.

  \begin{itemize}
  \item Autores: Daniel Adler, Duncan Murdoch, e outros
  \item Lançamento: 04-Mar-2004
  \item Versão: 0.95.1247
  \item URL: \url{http://cran.r-project.org/web/packages/rgl/index.html}
  \item Algumas aplicações com o \texttt{rgl}:

  \begin{itemize}
    \itemsep1pt\parskip0pt\parsep0pt
  \item \href{http://www.r-bloggers.com/?s=rgl}{Busca no R Bloggers}
  \end{itemize}
\end{itemize}

\end{frame}

%----------------------------------------------------------------------

\subsection*{Como usar}

\begin{frame}

\input{./tikz/ex_rgl1.tex}

\end{frame}

%----------------------------------------------------------------------

\subsection*{Exemplos}

\begin{frame}

 Praticando:
  \begin{enumerate}
  \item
    \href{run:./R/rgl/rgl.R}{R Script rgl}
  \item 
	\href{run:./rgl/RGL.html}{Galeria rgl IGUIR}
  \end{enumerate}

  \vspace{0.5cm}
  Algumas aplicações com o rgl:
  \begin{itemize}
  \item \href{http://cran.r-project.org/web/packages/rgl/vignettes/}{Galeria
      do autor},
  \item \href{http://www.r-bloggers.com/?s=rgl}{Busca no R
      Bloggers}
  \end{itemize}

\end{frame}

\section{\texttt{googleVis}}

%----------------------------------------------------------------------

\subsection{Descrição}

\begin{frame}

  \begin{itemize}
  \item Interface R para gráficos \textit{a la} Google Docs SpreadSheets, usa a
    \emph{Google Charts API}.
  \item O mais conhecido desses \textit{charts} é provavelmente o
    \textbf{Motion Chart}, popularizado por Hans Rosling em seu
    \href{??}{TED talk}. INCLUIR UM PRINTSCREEN E O LINK.
  \item Visualizar dados em data frames com gráficos Google sem upload
    no Google Docs.
  \item O resultado é um html com funções JavaScript hopedadas pelo
    Google que é rederizado pelo navegador.
  \item Requer conexão, às vezes flash.
  \end{itemize}

  \begin{itemize}
  \item Autores: Markus Gesmann, Diego de Castillo, Joe Cheng
  \item Lançamento: 03-Dec-2010
  \item Versão: 0.5.9
  \item URL:
    \url{http://cran.r-project.org/web/packages/googleVis/index.html},
    \url{https://github.com/mages/googleVis}
  \end{itemize}
  
\end{frame}

\begin{frame}
  \begin{itemize}
  \item Dado estruturado em \texttt{DataTable}.
  \item Transforma \texttt{data.frame}s em objetos JSON.
  \item Usa o \texttt{RJSONIO} para gerar JSON.
  \item Na versão 0.5.9: COLOCAR AQUI A LISTA... E UM PRINT SCREEN BEM CHAMATIVO.
  \end{itemize}
\end{frame}

%----------------------------------------------------------------------

\subsection{Como usar}
  
\frame{
  \includegraphics{./tikz/googleVis_gvisHistogram-1.pdf}
}  

\frame{
  \includegraphics{./tikz/googleVis_gvisHistogram-2.pdf}
}

\frame{
  \includegraphics{./tikz/googleVis_gvisHistogram-3.pdf}
}

%----------------------------------------------------------------------

\subsection{Exemplos}

\begin{frame}

 Praticando:
  \begin{enumerate}
  \item \href{run:./R/googleVis/googleVis.R}{R Script googleVis}
  \item \href{run:googleVis.html}{Galeria googleVis IGUIR}
  \end{enumerate}

  \vspace{0.5cm}
  Algumas aplicações com o googleVis:
  \begin{itemize}
  \item \href{http://cran.r-project.org/web/packages/googleVis/vignettes/}{Galeria
      do autor},
  \item \href{http://www.r-bloggers.com/?s=googleVis}{Busca no R
      Bloggers}
  \end{itemize}

\end{frame}

\section{\texttt{gWidgets}}

%----------------------------------------------------------------------

\subsection{Descrição}

\begin{frame}

  \texttt{gWidgets} fornece um kit de ferramentas para construir
  interfaces gráficas interativas.

  \vspace{\baselineskip}

  Verzani, M., Lawrence, M. (2012). \emph{Programming Graphical User
    Interfaces in R}, CRC Press.

  \begin{itemize}
  \item Autor: John Verzani
  \item Lançamento: 29-Sep-2006
  \item Versão: 0.0-54
  \item URL: \url{http://cran.r-project.org/web/packages/gWidgets/index.html}
  \end{itemize}

\end{frame}

\begin{frame}

  \begin{itemize}
  \item Alguns pacotes baseados em toolkits suportadas:
    \begin{itemize}
    \item tcl/tk:
      \href{http://cran.r-project.org/web/packages/gWidgetstcltk/index.html}{\texttt{gWidgetstcltk}}
      \href{http://cran.r-project.org/web/packages/Rcmdr/index.html}{\texttt{Rcmdr}},
      \href{http://cran.r-project.org/web/packages/TeachingDemos/index.html}{\texttt{TeachingDemos}},
      \href{http://cran.r-project.org/web/packages/MetSizeR/index.html}{\texttt{MetSizeR}},
      \href{http://cran.r-project.org/web/packages/MergeGUI/index.html}{\texttt{MergeGUI}},
      \href{http://cran.r-project.org/web/packages/GrapheR/index.html}{\texttt{GrapheR}},
      \href{http://cran.r-project.org/web/packages/BiplotGUI/index.html}{\texttt{BiplotGUI}},
      \href{http://cran.r-project.org/web/packages/TestScorer/index.html}{\texttt{TestScorer}}
      e muitos outros.
    \item gtk:
      \href{http://cran.r-project.org/web/packages/gWidgetsRGtk2/index.html}{\texttt{gWidgetsRGtk2}}
      \href{http://cran.r-project.org/web/packages/playwith/index.html}{\texttt{playwith}},
      \href{http://cran.r-project.org/web/packages/MissingDataGUI/index.html}{\texttt{MissingDataGUI}},
      \href{http://cran.r-project.org/web/packages/GroupSeq/index.html}{\texttt{GroupSeq}},
      \href{http://cran.r-project.org/web/packages/AtelieR/index.html}{\texttt{AtelieR}},
      \href{http://cran.r-project.org/web/packages/vmsbase/index.html}{\texttt{vmsbase}},
      \href{http://cran.r-project.org/web/packages/reshapeGUI/index.html}{\texttt{reshapeGUI}},
      \href{http://cran.r-project.org/web/packages/R2STATS/index.html}{\texttt{R2STATS}}
      e muitos outros.
  \end{itemize}
\end{itemize}

\end{frame}

%----------------------------------------------------------------------

\subsection{Como usar}

\begin{frame}

  modelo característico.

\end{frame}

%----------------------------------------------------------------------

\subsection{Exemplos}

\begin{frame}
  \begin{enumerate}
  \item \href{run:./tikz/code.Rnw}{code.Rnw}
  \item \href{./tikz/code.Rnw}{code.Rnw}
  \item \url{./tikz/code.Rnw}
  \item links para executáveis e galeria.
  \end{enumerate}
\end{frame}


\section{\texttt{rpanel}}

\subsection*{Descrição}

%----------------------------------------------------------------------

\begin{frame}

  \texttt{rpanel} fornece um conjunto de funções para criar interfaces
  gráficas simples para controlar funções do R. As interfaces são
  contruídas usando o pacote \texttt{tcltk}. Além das funções que
  fornecem elementos de interface, o pacote tem funções para interfaces
  específicas chamadas de \emph{cartoons}.

  \begin{itemize}
  \item Autores: Bowman, Bowman, Gibson and Crawford
  \item Lançamento: 21-Aug-2006
  \item Versão: 1.1-3
  \item URL:
    \url{http://cran.r-project.org/web/packages/rpanel/index.html}
\end{itemize}

\end{frame}

\begin{frame}

\begin{itemize}
\item Alguns pacotes baseados em \texttt{rpanel}:
  \href{http://cran.r-project.org/web/packages/GUIDE/index.html}{\texttt{GUIDE}},
  \href{http://cran.r-project.org/web/packages/MDSGUI/index.html}{\texttt{MDSGUI}},
  \href{http://cran.r-project.org/web/packages/RVideoPoker/index.html}{\texttt{RVideoPoker}},
  \href{https://github.com/walmes/wzRfun/blob/master/R/rp.nls.R}{\texttt{wzRfun::rp.nls}}.
  e mais.
\end{itemize}

\end{frame}

%----------------------------------------------------------------------

\subsection*{Como usar}

\begin{frame}
\includegraphics[scale=1]{./tikz/hist_slider_rpanel-1.pdf}
\end{frame}

\begin{frame}
\includegraphics[scale=1]{./tikz/hist_slider_rpanel-2.pdf}
\end{frame}

%----------------------------------------------------------------------

\subsection*{Exemplos}

\begin{frame}
 Praticando:
  \begin{enumerate}
  \item
    \href{run:./R/rpanel/rpanel.R}{R Script rpanel}
  \item 
	\href{run:rpanel.html}{Galeria rpanel IGUIR}
  \end{enumerate}

  \vspace{0.5cm}
  Algumas aplicações com o rpanel:
  \begin{itemize}
  \item \href{http://www.stats.gla.ac.uk/~adrian/rpanel/}{Galeria
      do autor},
  \item \href{http://www.r-bloggers.com/?s=rpanel}{Busca no R
      Bloggers}
  \end{itemize}

\end{frame}

\section{\texttt{shiny}}

%----------------------------------------------------------------------

\subsection{Descrição}

\begin{frame}

  \texttt{shiny} torna incrivelmente fácil construir aplicações web
  interativas com o R. Ligação entre \emph{inputs} e \emph{outputs} que
  são reativos e um conjunto extenso de \emph{widgets} permitem
  construir interfaces atraentes, responsivas e poderosas para a web com
  esforço mínimo.

  \begin{itemize}
  \item Autores: Winston Chang, Joe Cheng, JJ Allaire, Yihui Xie,
    Jonathan McPherson, e muitos contribuidores
  \item Lançamento: 01-Dec-2012
  \item Versão: 0.12.1
  \item URL:
    \url{http://cran.r-project.org/web/packages/shiny/index.html},
    \url{http://shiny.rstudio.com/}
  \end{itemize}

\end{frame}

\begin{frame}

  \begin{itemize}
  \item Algumas aplicações em \texttt{shiny}:
  \begin{itemize}
    \itemsep1pt\parskip0pt\parsep0pt
  \item \href{http://shiny.rstudio.com/gallery/}{Galeria Shiny Oficial},
  \item \href{http://www.showmeshiny.com/}{Galeria Shiny},
  \item \href{http://www.stat.cmu.edu:3838/hseltman/LogReg/}{Logistic
      Regression Residual Analysis},
  \item
    \href{https://hseltman.shinyapps.io/QuantileNormal}{Investigation of
      Quantile-Normal Plots Through Simulation},
  \item
    \href{http://www.stat.cmu.edu:3838/hseltman/PrePost/}{Pre-test/Post-test
      Simulation},
  \item
    \href{http://www.stat.cmu.edu:3838/hseltman/TransferFunctions/}{Explore
      Transfer Functions},
  \item
    \href{http://nbcgib.uesc.br/lec/avale-es/amb-virtual/inferencia/anava}{Fundamentos
      da análise de variância},
  \item
    \href{http://nbcgib.uesc.br/lec/avale-es/amb-virtual/probabilidade/con-frequentista}{Conceito
      frequentista de probabilidade}.
  \end{itemize}
\end{itemize}

\end{frame}

%----------------------------------------------------------------------

\subsection{Como usar}

\begin{frame}

modelo característico.

\end{frame}

%----------------------------------------------------------------------

\subsection{Exemplos}

\begin{frame}

links locais e galerias.

\end{frame}
\section{Ficaram de fora}

\begin{frame}

  \begin{itemize}
  \item \href{http://ramnathv.github.io/rCharts/}{\texttt{rCharts}}
  \item
    \href{http://cran.r-project.org/web/packages/iplots/index.html}{\texttt{iplots}}
  \item
    \href{http://cran.r-project.org/web/packages/rggobi/index.html}{\texttt{ggobi}}
  \end{itemize}

\end{frame}

\section{Considerações finais}

\begin{frame}

  \begin{itemize}
  \item 1
  \item 2
  \item \ldots{}
  \end{itemize}

\end{frame}

\section{Agradecimentos}

\begin{frame}

  \begin{itemize}
  \item A organização da RBRAS
  \item A comunidade R e do software livre
  \item LEG e PET
  \item Acadêmicos do curso de estatística
  \item Ao Prof.~Dr.~Ivan Allaman (UESC)
  \end{itemize}

\end{frame}

\end{document}
