\section{\texttt{rgl}}

\subsection{Descrição}

%----------------------------------------------------------------------

\begin{frame}

  \begin{itemize}
  \item \texttt{rgl} é uma biblioteca de funções para visualização
    interativa de gráficos tri-dimensionais (3D).
  \item Funções inspiradas nas 2D, de primitivas à médio e alto nível.
  \item Representações em 3D de gráficos e de objetos geométricos
    (cubos, elipses, etc).
  \item A visualização em tela com OpenGL, em HTML com WebGL.
  \end{itemize}

  \begin{itemize}
  \item Autores: Daniel Adler, Duncan Murdoch, e outros.
  \item Lançamento: 04-Mar-2004.
  \item Versão: 0.95.1247.
  \item URL: \url{http://cran.r-project.org/web/packages/rgl/index.html}.
\end{itemize}

\end{frame}

%----------------------------------------------------------------------

\subsection{Como usar}

\frame{
  \input{./tikz/ex_rgl1.tex}
}

\frame{
  \includegraphics{./tikz/rgl-2.pdf}
}

%----------------------------------------------------------------------

\subsection{Exemplos}

\begin{frame}

  Praticando:
  \begin{enumerate}
  \item \href{run:./R/rgl/rgl.R}{R Script rgl}
  \item \href{run:./rgl/RGL.html}{Galeria rgl IGUIR}
  \end{enumerate}

  Algumas aplicações com o rgl:
  \begin{itemize}
    \itemsep1pt\parskip0pt\parsep0pt
  \item
    \href{http://cran.r-project.org/web/packages/rgl/vignettes/}{Galeria
      do autor},
  \item \href{http://www.r-bloggers.com/?s=rgl}{Busca no R Bloggers}
  \end{itemize}

\end{frame}
